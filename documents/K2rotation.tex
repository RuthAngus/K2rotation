\documentclass[useAMS, usenatbib, preprint, 12pt]{aastex}
\usepackage{cite,natbib}
\usepackage{epsfig}
\usepackage{cases}
\usepackage[section]{placeins}
\usepackage{graphicx, subfigure}
\usepackage{color}

\title{Systematics-insensitive periodic signal search with K2}

\begin{document}

\newcommand{\article}{article}
\newcommand{\nGO}{4923}
\newcommand{\uHz}{$\mu$Hz}
\newcommand{\oxford}{1}
\newcommand{\nyu}{2}
\newcommand{\cfa}{3}

\author{%
   Ruth Angus\altaffilmark{\oxford},
   Daniel Foreman-Mackey\altaffilmark{\nyu},
   John A. Johnson\altaffilmark{\cfa},
   {\it et al.}
}

\altaffiltext{\oxford}{Subdepartment of Astrophysics, University of Oxford, OX1 3RH, UK}
\altaffiltext{\nyu}{Center for Cosmology and Particle Physics, New York University, NY, USA}
\altaffiltext{\cfa}{Harvard-Smithsonian Center for Astrophysics, 60 Garden St.,
Cambridge, MA, USA}

\begin{abstract}

% Aims
From pulsating stars to transiting exoplanets, the search for periodic signals
in {\it K2} data is relevent to a long list of scientific goals.
Systematics affecting {\it K2} light curves due to the decreased
spacecraft pointing precision inhibit the easy extraction of periodic signals
from the data.
We here develop a method for producing periodograms of K2 light curves that
are insensitive to pointing-induced systematics; the Systematics-Insensitive
Periodogram (SIP).
% Methods
Traditional sine-fitting periodograms use a generative model to find the
frequency of the sinusoid that best describes the data.
We extend this principle by including systematic trends, based on a set of
`Eigen light curves', following \citet{Foreman-Mackey2015}, in our generative
model as well as a sum of sine and cosine functions over a grid of
frequencies.
% Results
Using this method we are able to produce periodograms, free from
systematic features.
This method performs well for signals with periods from 1 hour up to 2 days,
however it is not optimized for periods much greater than 5 days.
This is due to the fact that some common trends between stars vary on
timescales similar to rotation periods.
The quality of the resulting periodograms, at short periods, are such that we
can recover acoustic oscillations in giants and detect variable stars,
eclipsing binaries and exoplanet candidates without the need for any
detrending.

\end{abstract}

\section{Introduction}
\label{Introduction}

The excellent precision achieved by the original {\it Kepler} mission relied
on extremely precise pointing, for which three reaction wheels were required.
% John: cite original mission and Howell here.
After the failure of one of these wheels, the {\it Kepler} team devised a new
pointing scheme where the spacecraft is stabilised by the Solar wind, so long
as it points at fields in the ecliptic plane \citep{Howell2014}.
In this configuration the spacecraft is able to maintain an unstable
equilibrium, with the two functioning reaction wheels controlling pitch and
yaw whilst the spacecraft slowly rolls about the boresight.
The spacecraft fires its thrusters once every $\sim$ 6 hours
\citep{Vanderburg2014} to correct for
this slow drift and, as stars move across pixels with different sensitivities,
their flux varies.
Developing methods for the extraction of high-precision photometry from K2
light curves, despite the reduced pointing precision, is extremely
important.
Several methods for detrending {\it K2} light curves have already been
developed.
For example, \citet{Vanderburg2014} (hereafter VJ14) and \citet{Crossfield2015}
use simple aperture photometry and correct the light curve of each star
individually and \citet{Aigrain2015} use a Gaussian process to model the
non-linear dependence of stellar flux on the roll angle of the telescope.
Whilst these methods successfully remove most systematic trends and
produce light curves suitable for exoplanet search and some stellar
variability studies, the $\sim$ 6 hour thruster firing signal may still appear
with high power in the periodograms of these detrended light curves.
Any physical processes with characteristic timescales of order
6 hours, asteroseimic oscillations in giants and short period
exoplanets, for example, therefore need to be treated extremely carefully.
 detrending methods for {\it K2} light curves, specifically intended for the
asteroseismic analysis of giant stars is being developed by Lund {\it et al.}
(in prep.), in which the systematics due to roll are corrected, again on
a star-by-star basis and any remaining periodic signals at 47 $\mu$Hz (6 hour
period) or its harmonics are removed by prewhitening.
The method developed here, the Systematics-Insenstive Periodogram (SIP)
produces a periodogram of the data without the need for detrending or
prewhitening.

As well as discovering thousands of exoplanets, the original {\it Kepler}
mission revolutionized many fields of stellar astronomy, particularly
asteroseismology.
Fundamental stellar parameters---in some cases, extremely precise ones---can
be calculated for {\it Kepler} asteroseismic stars from the power spectra of
their light curves.
Although Sun-like stars oscillate at high frequencies and require
short-cadence observations, pulsations of giant stars lie below the Nyquist
frequency set by the 28.5 minute sampling rate of long cadence {\it Kepler}
data: 283 $\mu$Hz.
Asteroseismic analysis of data from the original {\it Kepler} mission is
traditionally conducted upon detrended light curves.
For short cadence {\it Kepler} data, this detrending method is described in
\citet{Garcia2011}.
Due to the precise pointing of the original {\it Kepler} mission, systematics
present in these light curves, caused by temperature fluctuations and minor
pointing shifts, are relatively low amplitude.
This is not the case for {\it K2} light curves however: the precision over a
6 hour timescale is estimated to be 4 times worse in {\it K2} data
\citep{Howell2014}, therefore new approaches to the treatment of systematics
are necessary.
Figure \ref{fig:raw} demonstrates the need for careful systematics treatment
of {\it K2} photometry for asteroseismology.
The top panel shows a Lomb-Scargle (LS) periodogram of the raw, simple aperture
photometry\footnote{The method used to extract this photometry is described in
\textsection \ref{sec:Method}} of EPIC 201183188, a pulsating giant star.
The large peaks at $\sim$ 47 $\mu$Hz and its harmonics are caused by the
regular thruster fires of the spacecraft.
The bottom panel shows the LS periodogram of this light curve, after
it has been detrended using the method of VJ14.
The large peaks are still present in the detrended light curve.
While this remaining noise source does not interfere with the detection of
high-signal-to-noise transit events for periods greater than $\sim$1 day
\citep{Vanderburg2015}, it does hamper the detection of
smaller signals, particularly on time scales comparable to that of thruster
fires.
These peaks lie in an important region of parameter space for giant star
asteroseismology and could affect the stellar parameters measured for thousands
of giants if not dealt with appropriately.

Stellar rotation is another area of astronomy that has hugely benefitted from
the era of high-precision space photometry.
Active regions on the surface of rotating stars produce periodic variations
in flux---stellar rotation periods can therefore be measured from
{\it Kepler} light curves.
Stellar rotation is a field of active interest as the rotation period of star
can be used to infer its age \citep{Skumanich1972, Barnes2007, Epstein2014},
is thought to be tied to the stellar magnetic dynamo, and could even reveal
dynamical interations with companion stars or planets \citep[e.g.][]{Beky2014,
Poppenhaeger2014}.
Current methods for measuring rotation periods from {\it Kepler} light curves
include periodogram \citep[e.g.][]{Reinhold2013}, AutoCorrelation Function
(ACF) \citep{McQuillan2013} and wavelet \citep[e.g.][]{Garcia2014} analysis,
or some combination thereof.
Stellar variability is not typically sinusoidal, therefore sine-fitting
periodograms are not perfectly suited to measuring rotation periods.
For this reason, the ACF method is often favoured over the periodogram method.
However, because autocorrelation is performed directly on the light curves,
and cannot be written down as a generative model, it is not possible use
autocorrelation techniques on untreated {\it K2} data.
A quasi-periodic Gaussian process is a much better effective model for stellar
variability than a sinusoid, however we choose to focus on the more generally
applicable (and computationally tractable) sine-wave periodogram, leaving the
Gaussian process periodogram for future consideration.

In this \article\ we focus on the examples of asteroseismology and stellar
rotation, however many other fields of astronomy utilize periodic information
in {\it K2} light curves.
These include eclipsing binaries, variable stars exoplanets, white dwarfs and
even AGN.
The development of tools for extracting periodic information from {\it K2}
data is essential if it is to be as revolutionary in time-domain
astronomy as the original {\it Kepler} mission was.

\section{Method}
\label{sec:Method}

The method implemented in this \article\ is an extention of the planet-search
algorithm developed by \citet{Foreman-Mackey2015} (hereafter FM15).
All targets observed by {\it Kepler} move on the CCD in the same way,
therefore the systematics affecting each individual star's light curve have
shared properties.
The FM15 method uses this fact by decomposing the light curves into a set
of `Eigen Light Curves' (ELCs) using Principle Component Analysis (PCA), which
can be used to model any individual star's light curve with very little loss
of information.
This process is similar to the method used to produce PDC-MAP data for the
original {\it Kepler} mission \citep[][]{Stumpe2012, Smith2012}.
The resulting ELCs from campaign 1 can be used to model any campaign 1 {\it
K2} light curve, (campaign 0 ELCs for campaign 0, etc) and specifically, can
model the data in conjunction with an arbitrary physical model.
In order to construct sets of ELCs for campaigns 0 and 1, FM15 downloaded the
target pixel files for all stars in these two fields.
% ; 21,703 in total.
The position of each star was predicted using the World Coordinate System (WCS)
and 10 circular apertures placed around the star with radii varying from 1 to
5 pixels in steps of 0.5 pixels.
Following the procedure of VJ14, the aperture producing the
light curve with the lowest CDPP within a 6 hour window
\citep{Christiansen2012} was selected\footnote{The simple aperture photometry
light curves for campaigns 0 and 1 are available at
http://bbq.dfm.io/ketu/lightcurves/}.
PCA was then performed on the full set of targets in order to produce ELCs.
FM15 used 150 of these ELCs, plus a transit model, in order to
search for exoplanet candidates without the need for a `detrending' step.
The likelihood of the data, conditioned on the ELC-plus-transit
model was calculated over a fine grid of periods and transit depths, resulting
in the detection of 36 new exoplanet candidates.
We use exactly the same technique to find periodic signals in {\it K2} data,
however, instead of an exoplanet transit, our model is a sinusoid.
This model is linear, therefore the likelihood function conditioned on
a specific frequency can be calculated and the systematics model marginalized
over analytically.

Following the notation in FM15, our model for the $k$th star can be written
\begin{equation}
	\mathbf{f_k} = \mathbf{A}\mathbf{w_k} + \mathrm{noise},
\end{equation}
where $\mathbf{f_k}$ is the vector of $N$ flux values,
\begin{equation}
	\mathbf{f_k} = (f_{k,1}, f_{k,2}, f_{k,3}, ..., f_{k,N})^T
\end{equation}
at times
\begin{equation}
	\mathbf{t_k} = (t_1, t_2, t_3, ..., t_N)^T.
\end{equation}
$\mathbf{A}$ is the design matrix:
\begin{eqnarray}
	\mathbf{A} &=& \left (\begin{array}{ccccccc}
	x_{1,1} & x_{2,1} & \cdots & x_{150,1} & 1 & \sin(2\pi\nu t_1) & \cos(2\pi\nu t_1) \\
	x_{1,2} & x_{2,2} & \cdots & x_{150,2} & 1 & \sin(2\pi\nu t_2) & \cos(2\pi\nu t_2\\
    && \vdots &&&\\
	x_{1,N} & x_{2,N} & \cdots & x_{150,N} & 1 & \sin(2\pi\nu t_N) & \cos(2\pi\nu t_N)
\end{array}\right )
\end{eqnarray}
where the $x_{ij}$s are the ELCs\footnote{Campaign 0 and 1 ELCs are
available at http://bbq.dfm.io/ketu/elcs/}, with $i$ denoting the ELC number
and $j$ the time index.
The design matrix contains the basis functions of the linear model.
The basis functions for the systematic features in the light curves are the ELC
values at each time index, the sine and cosine terms are the basis functions of
the sinusoidal signal of interest, and the `1's describe a linear offset, or
`bias'.
Any {\it K2} light curve can be reproduced as a linear combination of these
basis functions.
We are interested in the last two elements of the weight vector: the
coefficients of the sinusoidal signal.
The maximum likelihood solution for the weight vector,
$\mathbf{w}$ is
\begin{equation}
	\mathbf{w_k}^* \gets (\mathbf{A}^T\mathbf{A})^{-1}\mathbf{A}^T\mathbf{f_k},
\end{equation}
and the sum of the last two elements of $\mathbf{w}_k$, squared, provides the
power of the signal at a given frequency.
This operation is performed over a grid of frequencies to produce a power
spectrum.
We then calculate the signal-to-noise ratio (S/N) frequency by comparing the
power at each frequency to a robust estimate of the variance in power across
all frequencies.
In practise, the S/N, rather than the power is plotted against frequency to
produce an SIP\footnote{The code used in this project can be found at
https://github.com/RuthAngus/SIPK2.}.

\subsection{Tests}

In order to test the sensitivity and completeness of the SIP method,
we injected sinusoids into real {\it K2} light curves.
We identified EPIC 201311941 as a quiet star with no significant periodic
variability at any frequency.
10,000 sinusoids with periods between 1 to 50 days and amplitudes ranging from
1 to 1000 parts-per-million (ppm) (evenly in log-space) were injected into
the raw campaign 1 light curve of EPIC 201311941.
The highest peak in the resulting SIP of each light curve was recorded.
If the period of the highest peak lay within 10\% of the true period, it was
considered a successful detection.
This procedure was repeated for periods ranging from 1 hour to 2 days in order
test the recovery of signals at frequencies relevent to asteroseismology,
and for periods between 2 and 30 days to test rotation period timescales.
The resulting completeness maps are shown in figures \ref{fig:K2_hist_r} and
\ref{fig:K2_hist_a}.
The SIP is uniformly sensitive across all frequencies relevant to giant
asteroseismology and is able to recover signals with amplitudes of a few ppm.
Figure \ref{fig:K2_hist_a} shows the lower sensitivity of the SIP for periods
greater than $\sim$ 2 days.
A caveat of these injection tests however, is that a single sinusoid with
a constant amplitude over the full 80 day baseline of campaign 1 may not
necessarily be a realistic model of acoustic modes in giant stars and is
certainly not representative of stellar variability.
Real signals, therefore are likely to be less easy to detect.

In order to demonstrate the consistent ability of the SIP method
to remove the signal at 47\uHz, corresponding to the periodic $\sim$6 hour
thruster firings, we computed SIPs for \nGO\ targets from the GO1049
proposal: ``Galactic Archaeology on a grand scale" (PI: Stello, D.).
For each target, an SIP and a LS periodogram of its
VJ14 light curve was calculated for frequencies between
44 and 54 \uHz.
Both the height and frequency of the highest peak in the SIP and the highest
peak in the LS periodogram was recorded.
A histogram of the frequencies of the highest peaks in the SIPs of all \nGO\
targets is shown in the top panel of figure \ref{fig:sip_hist}.
The bottom panel shows the histograms of peak heights within the
correspondingly colored ranges indicated in the top panel.
This figure shows that whilst there are a greater number of maximum peaks
around 47 \uHz, the S/N of these peaks are comparable to those found just
above and just below this frequency.
Figure \ref{fig:vbg_hist} shows the equivalent results for the
VJ14 light curves, however, in this case the bin heights are
plotted on a logarithmic scale.
There is a significant number of large peaks at $\sim$47 \uHz\ in the LS
periodograms of the detrended light curves; the highest peak in the LS
periodograms was almost always located at $\sim$ 47 \uHz.
Furthermore, the distribution of peak power within the range 46.5-48 \uHz\ is
skewed towards higher powers, i.e. a substantial fraction of the peaks at
$\sim$ 47 \uHz\ have a large power.

\subsection{Application to real light curves: asteroseismology}

An example LS periodogram\footnote{All LS periodograms
produced in this project were calculated using the gatspy Python module:
https://github.com/astroML/gatspy/tree/master/gatspy/periodic} of the raw {\it
K2 photometry} for giant star, EPIC 201211472 is shown in figure
\ref{fig:raw}.
Peaks appearing at 47 $\mu$Hz and its harmonics are produced by the regular
$\sim$ 6 hour thruster fires that repoint the spacecraft.
These peaks are also present in periodograms of the VJ14 detrended light
curves.
The presence of systematic signals at these timescales are problematic for
asteroseismic analysis since they lie in a region of frequency space
that is often populated by giant asteroseismic modes.
It is possible to remove these signals by `prewhitening' the data, i.e.
subtracting a sinusoid of that frequency from the data, however this process
will artificially supress all signals, both systematic and astrophysical, at
that frequency.
The SIP method eliminates the necessity for any such procedure.

In order to search for high signal-to-noise asteroseismic modes in the giant
star candidates of GO1059, we searched for a power excess in the SIPs using the
method of \citet{Huber2009}: autocorrelation functions were calculated for
sections of the power spectrum in order to search for regions of increased
correlation and locate the frequency of maximum power.
The increased correlation arises from the even frequency spacing of acoustic
modes, and the frequency of maximum correlation at the location of the power
excess corresponds to the large frequecy separation, $\Delta\nu$.
Figures \ref{fig:1} to \ref{fig:6} show example power spectra of 6 targets for
which we detect pulsations using this method.

% Our method does not perfectly model the systematics for every star.
% Of the power spectra generated for each of the n stars in GO1049, n of them
% showed power at 47$\mu$Hz.
% Examples of power spectra containing residual systematic features are shown in
% figure ?.

We also applied our analysis to 3 red giants and 3 $\delta$ Scuti stars
identified by Lund {\it et al.} (in prep), in campaign 0.
Since campaign 0 was shorter, $\sim$ 25 days, the SIP method is not as
successful at modelling the systematics as for campaign 1.
% and fewer stars observed?
For this reason the 47$\mu$Hz peaks caused by the regular 6 hour thruster
firings appear, with relatively high amplitudes for 2 of the 6 targets and with
moderate amplitudes for another 2 of the 6.
The two most sucessful SIPs, for targets EPIC 202086286 and EPIC
202068435 are shown in figure \ref{fig:c0}.

\subsection{Application to real light curves: stellar rotation}
The top panel of figure \ref{fig:rotation_poster_child} shows the light curve
of an active, rotating star, EP201317002.
This light curve has been detrended using the method of
VJ14.
The middle and bottom panels show an ACF and LS periodogram of the
detrended light curve.
% The highest peaks in both the ACF and periodogram are located at around 10
% days.
Figure \ref{fig:K2_rotation_poster_child} shows the raw light curve of the
same target (top panel) with its SIP (bottom panel).
% calculated by modeling the data as
% a linear combination of ELCs and a sine and cosine function at a range of
% frequencies.
Each of these 3 methods measures a rotation period of around 10 days for this
target.
Figure \ref{fig:top5} shows the 5 ELCs with the highest weights for this star.
Most of these ELCs vary on timescales close to 10 days.
The light curve of EPIC 201317002 is therefore easily described by the ELCs
alone, without the need for an additional sinusoidal signal.
In fact, many of the ELCs have dominant sinusoidal variation, with periods
of a few days (the cause of this is still unclear).
For this reason the SIP method is not as well suited to stellar rotation as it
is to asteroseismology, which is concerned with higher frequency signals.
In order to thoroughly test the advantages and shortcomings of the SIP method
for measuring stellar rotation periods, it would be beneficial to inject
sinusoids into raw {\it K2} light curves, then to detrend them using, for
example the VJ15 method and to attempt to recover signals using a LS
periodogram, or similar.
However, it has been shown that the ACF method often performs better than
periodogram methods for measuring stellar rotation periods
\citep[][]{McQuillan2013, McQuillan2013b, Mazeh2015}.
In practise, the ACF method is likely to perform just as well if not better
than the SIP method for stars with relatively high-amplitude variability.
Whilst the SIP method may outperform ACF in the low signal-to-noise cases,
any `marginal' rotation period measurements should be treated with caution
unless a representative uncertainty is provided.
Neither ACF nor periodogram methods are equipped to provide such uncertainties.
As such, performing ACF on detrended light curves is likely to provide
sufficient results for stellar rotation.

\subsection{Further examples}
Figures \ref{fig:RRLyrae} and \ref{fig:EB} show the conditioned light curves
and SIPs of an RR Lyrae star, selected from Guest observing program GO1018 and
an Eclipsing Binary (EB), selected from the {\it K2} EB and variable star
catalogue of \citet{Armstrong2015}, respectively.
Figure \ref{fig:planet} shows the conditioned light curve and SIP of a
short-period planet candidate.
This planet has a period of around 0.4 days, short enough to be detectable
with a periodogram, as was demonstrated for a number of ultra-short
period {\it Kepler} exoplanets by \citet{Sanchis-Ojeda2014}.
The top panels of these three figures show the {\it K2} light curves of these
objects, conditioned on the highest S/N sinusoidal signal in the
periodograms.
The trends describing the data at the highest S/N period were subtracted from
the raw light curve to produce these light curves.

Photometric variability in dwarf stars on timescales less than 8 hours, often
known as flicker, has been linked to surface gravity
\citep[][]{Bastien2013, Kipping2014}.
The metrics used to quantify photometric variability include finding the range
in intensity, counting the number of zero crossings and calculating the
root-mean-square (RMS) of the light curve.
Although these features are related to signal processing, they are operations
performed on detrended light curves, not inferred from periodograms.
However, it may be possible to derive a property of the periodogram that scales
with the density or surface gravity of a star, for example, the mean excess
power at frequencies near those relevent to granulation timescales.

\section{Conclusions}
We demonstrate that modelling campaign 1 {\it K2} photometry as a linear
combination of 150 PCA components plus a sinusoid can produce beautiful
periodograms without the need for detrending.
We find that the 47 $\mu$Hz signal, generated by the spacecraft thruster
fires is not present in the vast majority of systematics-insensitive
periodograms for more than 4000 targets
selected from the {\it K2} guest observer program, GO1059, ``Galactic
Archaeology on a grand scale" (PI: Stello, D.).
The SIP method is highly successful for campaign 1 targets where the large
number of stars, observed for a baseline of 80 days ensures that most of the
systematics are captured in the ELCs and we anticipate that it will
be equally effective for the up-and-coming campaigns.
It is not, however, suitable for campaign 0 light curves, where the reduced
observational baseline means that the systematics are not fully captured by the
top 150 ELCs.

The successes of the SIP method are most apparent in the high frequency regime,
particularly in the region of frequency space relevent to the study of
asteroseismic oscillations in giant stars.
It performs well for signals with periods between $\sim$ 1 hour and $\sim$
2 days, however it is not well optimised for measuring longer periods, due to
the fact that many ELCs are dominated by periodic variations on the same
timescales as the rotation periods.

\begin{figure*}
\begin{center}
\includegraphics[width=6in, clip=true]{rawvbg_201183188.pdf}
\caption{LS periodogram of the raw (Top) and detrended
	 (bottom) {\it K2} photometry for EPIC 201183188. The light curve was
 	 detrended using the method of VJ14. The peak at
	 $\sim$ 47 $\mu$Hz and its harmonics produced by the regular spacecraft
	 thruster fires are still present in periodogram of the detrended
	 data.}
\label{fig:raw}
\end{center}
\end{figure*}

\begin{figure}
\begin{center}
	\subfigure[]{
            \label{fig:1}
	    \includegraphics[width=3in]{201545182.pdf}
        }
	\subfigure[]{
            \label{fig:2}
	    \includegraphics[width=3in]{201183188.pdf}
        }
	\subfigure[]{
            \label{fig:3}
	    \includegraphics[width=3in]{201211472.pdf}
        }
	\subfigure[]{
            \label{fig:4}
	    \includegraphics[width=3in]{201433687.pdf}
        }
	\subfigure[]{
            \label{fig:5}
	    \includegraphics[width=3in]{201444854.pdf}
        }
	\subfigure[]{
            \label{fig:6}
	    \includegraphics[width=3in]{201607835.pdf}
        }
    \end{center}
    \caption{SIPs of 6 long cadence {\it K2} giants with pulsations. These
	     were selected from the guest observing program, GO1059 and were
	     identified using the method of \citet{Huber2009}.
\label{fig:astero_examples}}
\end{figure}

\begin{figure}
\begin{center}
	\subfigure[]{
            \label{fig:c01}
	    \includegraphics[width=3in]{202086286.pdf}
        }
	\subfigure[]{
            \label{fig:c02}
	    \includegraphics[width=3in]{202068435.pdf}
        }
    \end{center}
    \caption{SIPs of 2 long cadence {\it K2} giants with
	    pulsations from campaign 0. These targets were identified as
	    pulsating giants by Lund {\it et al.} (in prep).
\label{fig:c0}}
\end{figure}

\begin{figure*}
\begin{center}
\includegraphics[width=6in, clip=true]{rotation_poster_child.pdf}
\caption{{\it Top}: Light curve of EPIC 201317002, detrended using the method
of VJ14. {\it Middle}: Autocorrelation function of the
detrended light curve. The autocorrelation function method measures a rotation
period of 8.77 days for this star. {\it Bottom}: The LS periodogram
of the detrended light curve. The highest peak in the periodogram is centred at
10.52 days.}
\label{fig:rotation_poster_child}
\end{center}
\end{figure*}

\begin{figure*}
\begin{center}
\includegraphics[width=6in, clip=true]{K2_rotation_201317002.pdf}
\caption{{\it Top}: Raw light curve of EPIC 201317002. {\it Bottom}: A
periodogram produced by modelling the data using the top 150 ELCs
plus a sine and cosine function at a range of frequencies. The highest peak in
the periodogram is centred at 10.34 days.}
\label{fig:K2_rotation_poster_child}
\end{center}
\end{figure*}

\begin{figure*}
\begin{center}
\includegraphics[width=6in, clip=true]{201317002_top5.pdf}
\caption{The 5 highest-weighted ELCs for EPIC 201317002. At least 3 of these
	ELCs show sinusoidal variability with a $\sim$ 10 day period.
	The light curve of EPIC 201317002 is therefore easily described by the
	ELCs alone, without the additional sinusoid model.
	This illustrates the difficulty in measuring rotation periods using the
	SIP method.}
\label{fig:top5}
\end{center}
\end{figure*}

\begin{figure*}
\begin{center}
\includegraphics[width=6in, clip=true]{K2_hist_r.pdf}
\caption{Completeness of sinusoidsal signal recovery. 10,000 Sinusoidal signals
with a range of periods and amplitudes were injected into the raw light curve
of the non-variable star EPIC 201311941. In the range of periods relevent to
stellar rotation ($\sim$ 1 - 60 days), we are most complete at the shortest
periods; the SIP is not well suited to measuring long periods.}
\label{fig:K2_hist_r}
\end{center}
\end{figure*}

\begin{figure*}
\begin{center}
\includegraphics[width=6in, clip=true]{K2_hist_a.pdf}
\caption{Completeness of sinusoidsal signal recovery. 10,000 Sinusoidal signals
with a range of periods and amplitudes were injected into the raw light curve
of the non-variable star EPIC 201311941. In the range of periods relevent to
asteroseismology ($\sim$ 1 - 48 hours), we are uniformly complete down to
very low amplitudes.}
\label{fig:K2_hist_a}
\end{center}
\end{figure*}

\begin{figure*}
\begin{center}
\includegraphics[width=6in, clip=true]{sip_hist.pdf}
\caption{{\it Top:} Histogram of the frequencies of the highest peaks in the
	SIPs of \nGO\ {\it K2} targets within the range 44 - 54 \uHz.
	{\it Bottom:} Histograms of peak heights within the correspondingly
	colored ranges indicated in the top panel.
	Whilst there is a larger number maximum peaks around 47 \uHz\ (the
	frequency corresponding to the 6 hour thruster fire) the amplitudes of
	these maximum peaks are comparable to the maximum peak heights just
	above and just below this frequency.}
\label{fig:sip_hist}
\end{center}
\end{figure*}

\begin{figure*}
\begin{center}
\includegraphics[width=6in, clip=true]{vbg_hist.pdf}
\caption{{\it Top:} Histogram of the frequencies of the highest peaks in the
	LS periodograms of the \citet{Vanderburg2014} light curves of \nGO\
	{\it K2} targets within the range 44 - 54 \uHz.
	{\it Bottom:} Histograms of peak heights within the correspondingly
	colored ranges indicated in the top panel.
	Note that the bin heights are on a logarithmic scale.
	The frequency of maximum peak height was $\sim$ 47 \uHz\ in almost
	every periodogram.
	Furthermore, the distribution of maximum peak height within the range
	46.5-48 \uHz\ is skewed towards higher powers, i.e. a large fraction of
	the peaks at $\sim$ 47 \uHz\ have a large power.
}
\label{fig:vbg_hist}
\end{center}
\end{figure*}

\begin{figure*}
\begin{center}
\includegraphics[width=6in, clip=true]{RR_201339783.pdf}
\caption{{\it Top:} The light curve of RR Lyrae star, EPIC 201339783,
	conditioned on the highest amplitude sinusoidal signal found in the
	SIP. {\it Bottom:} The systematic-insensitive periodogram of
	this light curve. This target was selected from GO1018.}
\label{fig:RRLyrae}
\end{center}
\end{figure*}

\begin{figure*}
\begin{center}
\includegraphics[width=6in, clip=true]{EB_201473612.pdf}
\caption{{\it Top:} The light curve of eclipsing binary, EPIC 201473612,
	conditioned on the highest amplitude sinusoidal signal found in the
	SIP. {\it Bottom:} The systematic-insensitive periodogram of
	this light curve. This target was selected from the catalogue of EBs
	and variable stars of \citet{Armstrong2015}.}
\label{fig:EB}
\end{center}
\end{figure*}

\begin{figure*}
\begin{center}
\includegraphics[width=6in, clip=true]{planet_201637175.pdf}
\caption{{\it Top:} The light curve of exoplanet candidate, EPIC 201637175,
	conditioned on the highest amplitude sinusoidal signal found in the
	periodogram. {\it Bottom:} The systematic-insensitive periodogram of
	this target.}
\label{fig:planet}
\end{center}
\end{figure*}

\bibliographystyle{plainnat}
\bibliography{k2rotation}
\end{document}
